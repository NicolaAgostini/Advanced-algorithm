\documentclass{article}
\usepackage[utf8]{inputenc}

\title{Laboratorio 1}
\author{Nicola Agostini, Roberto Cedolin, Lisa Parma}
\date{March 2019}

\usepackage[italian]{babel}
\usepackage[T1]{fontenc}
\usepackage{amsmath}
\usepackage{natbib}
\usepackage{graphicx}
\usepackage{placeins}
\usepackage{mathtools, nccmath}
\usepackage{hyperref}
\usepackage{biblatex}
\DeclarePairedDelimiter{\nint}\lfloor\rceil

\begin{document}

\maketitle

\section*{Introduzione}

In questo laboratorio si vuole analizzare la connettività di grafi diversi tra loro. In particolare viene utilizzato l'algoritmo UPA ed ER per la creazione di grafi in modo automatico e la lettura di un file di testo per la creazione di un grafo rappresentante una rete di calcolatori reale.
\section*{Domanda 1}
\begin{center}
\begin{figure}[h]
\includegraphics[scale=0.7, \textwidth, left]{Fig1_resilienze_attacchi_casuali.png}
\caption{Grafico domanda 1}
\label{fig:graph1}
\end{figure}
\end{center}

All'interno della figura \ref{fig:graph1} è rappresentato l'andamento della resilienza dei tre grafi (UPA, ER e dati reali) dopo la disattivazione di un numero crescente di nodi scelti casualmente. Come si può notare, il grafo che rappresenta i dati reali, a parità di nodi eliminati mostra generalmente una resilienza minore rispetto ai grafi generati con gli algoritmi ER ed UPA.\newline Il valore di p utilizzato per la creazione del grafo attraverso l'algoritmo ER è pari a 
\[
    \frac{12572}{\binom{6474}{2}}=0.000600
\]
questo perchè per avere un numero di archi pari a 12572 è necessario scegliere un arco ogni 6000 dal grafo completo. Quindi la probabilità di scegliere un arco è appunto 0.0006.
Il valore di m scelto è pari al grado medio dei vertici del grafo reale diviso per due. Il parametro m viene calcolato con 
\[
  \nint*{\frac{6474}{12572}}=2
\]
che è il grado medio dei nodi del grafo diviso 2.

\FloatBarrier

\section*{Domanda 2}
\begin{figure}[h]
\includegraphics[scale=0.7, \textwidth, left]{Fig2_resilienze_attacchi_casuali_masked.png}
\caption{Grafico domanda 2}
\label{fig:graph2}
\end{figure}
Nella figura \ref{fig:graph2} è presente l'andamento delle tre curve dei grafi dopo la rimozione di nodi in modo casuale. Inoltre sono presenti anche due rette, una verticale in posizione del 20\% dei nodi disattivati, l'altra che indica quando la dimensione della componente connessa più grande è superiore al 75\% del numero dei nodi ancora attivi.\\
In generale un grafo risulta essere resiliente quando il punto di intersezione con la retta verticale (rappresentante il 20\% dei nodi disattivati) è maggiore della retta inclinata che indica l'andamento resiliente.\\Tramite questa rappresentazione notiamo che tutti e tre i grafi risultano essere resilienti.
\FloatBarrier
\section*{Domanda 3}
\begin{center}
\begin{figure}[h]
\includegraphics[scale=0.7, \textwidth, left]{Fig3_resilienze_attacchi_intelligenti.png}
\caption{Grafico domanda 3}
\label{fig:graph3}
\end{figure}
\end{center}
Nel grafico in figura \ref{fig:graph3} vi è l'andamento dei grafi in seguito ad un attacco che ad ogni passo disattiva il nodo di grado massimo.\\A parità di nodi disattivati, il grafo ER mantiene una componente connessa maggiore rispetto al grafo UPA ed a quello dei dati reali.
Ciò è dovuto dal fatto che l'algoritmo ER collega casualmente i nodi in modo da avere una rete più resistente a questo tipo di attacchi.

\FloatBarrier

\section*{Domanda 4}
\begin{center}
\begin{figure}[h]
\includegraphics[scale=0.7, \textwidth, left]{Fig4_resilienze_attacchi_intelligenti_masked.png}
\caption{Grafico domanda 4}
\label{fig:graph4}
\end{figure}
\end{center}

Nella figura \ref{fig:graph4} si può notare che solamente il grafo ER è resiliente poichè il punto di intersezione tra la retta verticale (rappresentante il 20\% dei nodi disattivati) con la curva del grafo ER (curva blu) è al di sopra della retta che rappresenta l'andamento resiliente.
\\I rimanenti due grafi, invece, non sono resilienti poichè in seguito agli attacchi la dimensione massima delle componenti connesse è inferiore al 75\% dei nodi ancora attivi.

\FloatBarrier
\end{document}
